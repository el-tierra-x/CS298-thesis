\chapter{Background}
This section reviews the key literature and existing systems related to image security, certificate verification, and medical data protection through blockchain. It then illustrates how metadata hashing and smart contracts can be leveraged to create tamper-proof solutions.

\section{Related Work}

\subsection{Blockchain Technology and Metadata Hashing for Photo Integrity}
Image tampering is a widespread issue in digital media. With users posting images on dozens
of social sites, authentication becomes ever harder when traditional approaches depend on
readily modifiable metadata kept in centralized databases. Blockchain technology presents
an immutable, decentralized ledger initially envisioned by Nakamoto~\cite{nakamoto2008} as
a tamper-evident solution for protecting such metadata. By employing hashes of key image
characteristics, like timestamps, geospatial information, and device identifiers, and by
keeping them on-chain, any future alterations to the original image will cause a discrepancy,
thereby indicating possible tampering~\cite{cheng2020}.

Among the numerous systems utilizing this method, two of the most popular platforms are
particularly worth mentioning: PHOTOCHAIN and Go-Sharing. PHOTOCHAIN~\cite{wessely2023} is
a secure marketplace and community area that is particularly designed for photographers
and digital content creators. Its process is initiated when a user uploads an image
through PHOTOCHAIN's decentralized platform, which generates a metadata hash written to
the blockchain. Smart contracts manage licensing terms and ownership transfer, enabling
transparent and traceable handoffs whenever a piece is purchased or licensed. When an
image is uploaded and then someone attempts to tamper with it, the on-chain hash will no
longer match, making the tampering evident right away. PHOTOCHAIN also addresses this further
with an integrated payment module---usually cryptocurrency-based---for instant and trackable
compensation.

Similarly, Go-Sharing~\cite{zhang2023} tackles cross-platform photo sharing with a focus
on privacy. Photos are shared or uploaded in a system that utilizes both encryption and
metadata hashing for confidentiality and integrity. Smart contracts automate permission
checks to restrict decryption and viewing to whitelisted addresses. Any unauthorized tampering
with the content of a photo automatically causes a hash mismatch, which sends notifications
to content owners. Go-Sharing also focuses on easy-to-use monitoring capabilities, which allow
creators or owners to monitor continuously their content's authenticity status on social networks.

Through the utilization of publicly accessible ledgers instead of central servers, PHOTOCHAIN
and Go-Sharing establish greater degrees of user trust. 

Users can simply verify whether the
stored hash of an image is equal to its current state, eliminating the necessity of a trusted
third party to oversee or validate the process. Aside from the technological advantages, both
systems emphasize user-friendly experiences---a critical factor for the wider adaptation of
blockchain technology within the creative industries. Even as the pace of image manipulation
using AI keeps growing, these initiatives illustrate how secure hashing techniques, coupled
with automatic verification through smart contracts, can successfully curtail the spread of
manipulated or misrepresentative images.

\subsection{Blockchain for Certificate Verification}

Whereas digital photographs represent one domain susceptible to forgery, academic and
professional credentials are equally subject to such integrity threats. Classic authentication
of certificates depends heavily on manual crosschecks and centralized administrative authorities.

Not only does this delay the validation process, but it also creates weaknesses whereby
certificates can be forged easily. Blockchain-based systems bypass these limitations by storing
credential hashes on an immutable ledger. By checking these on-chain records, external entities
can instantaneously verify both the existence and validity of an asserted qualification.

Blockchain-Based Systems. Initiatives such as CERTbchain demonstrate the promise of decentralized
applications in making certificate issuance and verification more efficient~\cite{malsa2021}.
Whenever an institution issues a credential, a unique hash---based on the certificate's metadata---is
recorded on the blockchain. Employers or other verifying parties can then request the blockchain
to confirm that the certificate is complete and unaltered. This mechanism minimizes the need for
numerous intermediaries and can provide near-instant verification, providing considerable advantages
over conventional methods that may take several days or weeks to verify manually. Another model,
as discussed by Senthilkumar \cite{senthilkumar2023}, builds on the idea with the
addition of layered access controls and revocation procedures through the use of smart contracts.
This approach allows organizations to instantaneously revoke or alter certificates, hence ensuring
the most accurate records are present while maintaining historical chain-of-custody integrity.
As such, any manipulation or outdated credential would immediately be apparent during the
verification procedure.

This lessens the likelihood of fraud and promotes a more conducive environment for employers,
students, and institutions.

\subsection{Medical Image Sharing using Blockchain}
The healthcare field is one that highlights the imperative of data security and integrity,
particularly for medical imaging modalities like X-rays, CT scans, and MRIs. Any tampering with
such images could have serious consequences for patient care, leading either to misdiagnoses or
compromised treatment. Consequently, researchers and healthcare providers have looked to blockchain
technology to ensure that a record, once generated, cannot be modified without detection. In a sample
system, Tang. proposed a Medical Images Sharing System based on blockchain and smart
contract-based access control~\cite{tang2018}. In it, all the medical images are stored on-chain with
a unique cryptographic reference. Only authorized parties---specialists, physicians, or patients---are
granted permission to view or move these images, regulated by role-based smart contract logic.

If one tries to alter or substitute an image, the cryptographic hash that is stored on the blockchain
will no longer be the same, thereby instantly indicating tampering.

Wen built on this concept by proposing an Authenticable Medical Image-Sharing Scheme that
incorporates blockchain technology with an embedded QR code containing shadow information~\cite{wen2022}.
This hybrid system provides an additional layer of verifiability by combining on-chain records with
off-chain scan functionality.

Upon every scan of the QR code, the system re-calculates the image hash and compares it to the value
on the blockchain. In the event of tampering, discrepancies are immediately apparent. The solution
emphasizes the versatility of blockchain in being able to support a variety of cryptographic techniques,
noting secure data sharing and authenticatable validity in sensitive healthcare contexts.


\section{Blockchain Image Hashing and Metadata with Smart Contracts}
While the previous works focus on the benefits of having cryptographic pointers to data and images
on the blockchain, our solution differentiates by targeting full decentralization and user control
from the start---excluding centralized backends, special logins, or data stores per se.

\textbf{Store Only Hashes On-Chain (No IPFS or Batching Yet):}
In contrast to most systems which incorporate blockchain storage with off-chain support such as IPFS,
we have a straightforward trustless process where users determine where to store images (personal cloud
or local drive) and only store a cryptographic hash on Ethereum. Later versions can incorporate IPFS or
transaction batching to increase scalability and minimize expenses; nonetheless, our proof-of-concept
demonstrates the functional feasibility of a completely decentralized architecture without these
optimizations~\cite{chunmiao2021,andrade2022}.

\textbf{Decentralized Authentication:}
Instead of a backend login flow, users interact directly with the blockchain using Ethereum wallets.
Private keys are the only guardian of ownership, removing server-side credential management and
lessening security vulnerabilities~\cite{cheng2020}.

\textbf{On-Chain Verification:}
When a user shares or verifies an image, the app re-calculates its hash locally and compares it to the
on-chain reference. Discrepancies immediately expose unauthorized changes. This trustless, automated
process discourages any server-side manipulation or hidden changes.

\textbf{Fully User-Governed Storage:}
Whereas PHOTOCHAIN or Go-Sharing would be able to use proprietary backends or partially-centralized
storage, our proposal does not make any sacrifices with regards to user control. They have full
optionality in choosing how (and where) they wish to keep their image information; the blockchain
merely stores the metadata hash. The cost-effective architecture conserves on-chain storage space
while protecting the user's data from breaches of centralised storage.

\noindent{
\textbf{Differences between Go-Sharing and PHOTOCHAIN.}
}

Though we have similar objectives---such as inhibiting photo manipulation---our system differs from
current platforms in the following manners:

\begin{itemize}
  \item \emph{No Dedicated Backend:} We employ smart contracts exclusively and local/third-party
        storage of the user's preference. This reduces points of failure and encourages greater
        decentralization.
  \item \emph{Wallet-Only Authentication:} We eschew username/password systems; users authenticate
        through their Ethereum wallet, which is simpler to access and more private.
  \item \emph{Transparent On-Chain Actions:} The user interface shows explicit blockchain interactions
        (e.g., proving ownership or sending images), with every transaction and fee appearing in
        real-time.
  \item \emph{Self-Contained Verification:} All hash calculations are done on the client-side,
        precluding any tampering by a server process and permitting immediate detection of
        inconsistencies.
\end{itemize}

Although this design now rules out the incorporation of IPFS and sophisticated gas optimization
techniques (such as transaction batching and Layer 2 rollups), it indicates the viability of a
simple on-chain metadata hashing scheme. Additional optimizations for scalability and cost reduction
will be investigated in future work.