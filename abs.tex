In the age of the internet, the integrity of photographs is now undermined by the general availability of artificial intelligence (AI) technologies, such as deepfakes and generative adversarial networks (GANs), that enable the creation of realistic yet fake images. This has led to severe problems with misinformation and manipulation of digital information. The need for a trustworthy method of verifying the authenticity of photographs has become increasingly important.

This report presents a decentralized iOS application that addresses such concerns by ensuring the authenticity of photos with the support of blockchain technology. The application employs Ethereum-based smart contracts and cryptographic hashing to store photo metadata securely. When the user takes a photo, the application immediately hashes the image and essential metadata, such as the date, geolocation, and device details, and logs the resulting hash in the blockchain. This establishes a verifiable proof of the authenticity of the photo that cannot be altered or forged.

The authenticity of the image can later be verified by the users by re-uploading the image to the app. The system then recalculates the hash and compares it with the one saved in the blockchain, verifying if the image has been tampered with. The app also employs the use of Merkle Trees to support partial metadata validation, providing flexibility where some of the metadata may not be available.

The solution is very cost-effective to deploy. It is inexpensive—approximately 0.00025 Sep Eth per hash, or approximately \$0.0000326 USD. The operation itself is also quick and free for the users, with near-instantaneous verification with no fee, thanks to the optimized smart contract design. These findings show that the solution is feasible and scalable, offering a cost-effective approach to maintaining photo integrity with minimal cost.

{\bf Keywords:} Blockchain, Smart Contracts, Tamper-Proof Photos, Metadata Hashing, Medical Image Verification, Cryptographic Hashes.

